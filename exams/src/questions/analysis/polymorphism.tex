\titledquestion{Polymorphism}
\droptotalpoints
\begin{parts}
\part[6] 
Identify three examples of polymorphism in the following Java expression:
\begin{verbatim}
"1" + ((2 + 4) + 3.5) 
\end{verbatim}
Which kinds of polymorphism do they represent?
Explain the differences.

\begin{solution}
The \verb-+- operator is overloaded (ad-hoc polymorphism):
  It either performs a string concatenation (1st occurence), 
  an integer addition (2nd occurence),
  or a floating point addition (3rd occurrence).
These are two different kinds of overloading.
The first and second occurence perform different operations on different types,
while the second and third occurrence perform similar operations on different
types.

The result of \verb=(2 + 4)= is converted (implicit coercion, ad-hoc
polymorphism) from an integer value into a floating point value.
  Similarly, the result of \verb=((2 + 4) + 3.5)= is converted into a string value.

With operator overloading, the types of the operands determine which operation
is performed.
With type coercion, there is no operation that can handle the operand types,
i.e. there is no operation that can handle an integer and a floating point
operand or a string and a floating point operand. 
Instead, the operand is converted to a type an
operation can handle.
\end{solution}

\part[4]
Explain the difference between method overloading and method overriding.
Illustrate your explanation with an example in Java.

\begin{solution}
Method overloading and overriding is about different methods with the same name.
%
Method overloading can take place in unrelated classes, in the same class, and
in classes related by inheritance.
In unrelated classes, there are no constraints on the parameter or return types
of the overloaded methods (e.g. \Verb+A.m()+ and \Verb+B.m()+ or \Verb+A.m()+
and \Verb+B.m(A)+).
In the same class, the parameter types of
methods with the same name need to be different (e.g. \Verb+B.m()+ and
\Verb+B.m(A)+).
The same holds for classes related by inheritance (e.g. \Verb+B.m()+ and
\Verb+C.m(A)+).

In contrast, method overriding can only take place in classes related by
inheritance.
The method in the subclass needs to have the same and the same parameter types
(at least in Java) and a covariant return type (subtype of the overridden
method's return type).
In the example, \verb+C.m(A)+ overrides \verb+B.m(A)+ (same parameter types,
covariant return type) but overloads \verb+B.m()+ (different parameter types).

\fvset{frame=lines}
\fvset{framesep=8pt}
\begin{minipage}[t]{.3\textwidth}
\begin{Verbatim}
class A {
  public int m() {...}
}

\end{Verbatim}
\end{minipage}
\hfill
\begin{minipage}[t]{.3\textwidth}
\begin{Verbatim}
class B {
  public int m() {...}
  public B m(A a) {...}
}
\end{Verbatim}
\end{minipage}
\hfill
\begin{minipage}[t]{.3\textwidth}
\begin{Verbatim}
class C extends B {
  public C m(A a) {...}
}

\end{Verbatim}
\end{minipage}


\end{solution}

\end{parts}
