\titledquestion{Term rewriting}
\droptotalpoints

\emph{Stratego} provides a strategy \verb+inverse+ with the following
implementation:

\begin{center}
\begin{minipage}[t]{.6\textwidth} 
\begin{Verbatim}
inverse(|a): []     -> a
inverse(|a): [x|xs] -> <inverse(|[x|a])> xs 
\end{Verbatim}
\end{minipage}
\end{center}

\begin{parts}

\part[2]
Explain the semantics of \verb+inverse+ in English. 

\begin{solution}
\Verb+inverse+ rewrites a list to a new list.
The resulting list has the same elements as the original list, but in reversed order.
\end{solution}

\part[4]
What is the result of applying \Verb+inverse(|[])+ to the term \Verb+[1,2,3]+?
Show each step of computation.

\begin{solution}
\begin{minipage}[t]{.6\textwidth} 
\begin{Verbatim}
<inverse(|[])> [1,2,3] => <inverse(|[1])> [2,3] => 
<inverse(|[2,1])> [3] => <inverse(|[3,2,1])> [] => [3,2,1] 
\end{Verbatim}
\end{minipage}
\end{solution}

\part[4]
Based on the definition of \verb+inverse+, explain how an accumulator is used. 

\begin{solution}
An accumulator stores a temporary result of an ongoing computation.
\Verb+inverse+ uses its term parameter to accumulate the inverted list.
When the strategy is called, the accumulator should be initialised with the
empty list.
The second rule accumulates the inverted list by prepending the current head of
the list to the so far accumulated list.
It calls \Verb+inverse+ recursively on the tail of the list with the new
accumulator.
The first rule returns the accumulated list as the result of the inversion.
\end{solution}

\end{parts}
