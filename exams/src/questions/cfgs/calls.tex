\titledquestion{Formal grammars}
\droptotalpoints

Let $G_1$ be a formal grammar with nonterminal symbols $S$, and $P$,
terminal symbols '$\mathbf{f}$', '$\mathbf{x}$', '$\mathbf{,}$', '$\mathbf{(}$' and '$\mathbf{)}$', 
start symbol $S$, and the following production rules:
\begin{eqnarray*}
S & \rightarrow & \mathbf{f\ (\ } P \mathbf{\ )}\\
P & \rightarrow & \mathbf{x} \\
P & \rightarrow & P \mathbf{\ ,\ x} \\
P & \rightarrow & \mathbf{x\ ,\ } P 
\end{eqnarray*}

\begin{parts}

\part[1]
Is $G_1$ context-free? Why (not)?

\begin{solution}
Yes, because all production rules are of the form $N \times \left( N \cup \Sigma \right)^*$. 
\end{solution}

\part[2]
Describe the language defined by $G_1$ in English.

\begin{solution}
The language consists of all applications of a function symbol $\mathbf{f}$ to one or more parameters $\mathbf{x}$. 
Parameters are separated by commas and surrounded by parentheses.
\end{solution}

%\part[2]
%Use $G_1$ to generate a sentence with at least five words. 
%Show each generation step.
\part[3]
Give a left-most derivation for the sentence $\mathbf{f ( x , x , x )}$
according to $G_1$.

\begin{solution}
$S \Rightarrow 
\mathbf{f\ (\ } P \mathbf{\ )} \Rightarrow 
\mathbf{f\ (\ } P \mathbf{\ ,\ x\ )} \Rightarrow 
\mathbf{f\ (\ } P \mathbf{\ ,\ x\ ,\ x\ )}\Rightarrow 
\mathbf{f\ (\ x\ ,\ x\ ,\ x\ )}$
\end{solution}

\part[4]
Use $\mathbf{f ( x , x )}$ as an example to explain why $G_1$ is ambigous.

\begin{solution}
There are two different left-most derivations for the same word.

$S \Rightarrow 
\mathbf{f\ (\ } P \mathbf{\ )} \Rightarrow 
\mathbf{f\ (\ } P \mathbf{\ ,\ x\ )} \Rightarrow 
\mathbf{f\ (\ x\ ,\ x\ )}$
and 
$S \Rightarrow 
\mathbf{f\ (\ } P \mathbf{\ )} \Rightarrow 
\mathbf{f\ (\ } \mathbf{x\ ,\ } P \mathbf{\ )} \Rightarrow 
\mathbf{f\ (\ x\ ,\ x\ )}$
\end{solution}

\end{parts}
