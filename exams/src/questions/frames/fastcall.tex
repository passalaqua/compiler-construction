\titledquestion{Calling conventions}
\droptotalpoints

A compiler translates a function call and a function body to the following instructions for a register-based machine:

\begin{center}
\begin{minipage}[t]{5cm} 
\fvset{frame=lines}
\fvset{framesep=8pt}
\fvset{label=\textrm{function call}}
\begin{Verbatim}
    mov  AX 21
    mov  DX 42
    call _f@8
\end{Verbatim}
\end{minipage}
\hspace{1cm}
\begin{minipage}[t]{5cm}
\fvset{frame=lines}
\fvset{framesep=8pt}
\fvset{label=\textrm{function body}}
\begin{Verbatim}
    push BP
    mov  BP SP
    add  AX DX
    pop  BP
    ret
\end{Verbatim}
\end{minipage}
 \end{center}

\begin{parts}
\part[1]
Which calling convention do these instructions follow? 

\begin{solution}
FASTCALL
\end{solution}

\part[1]
What are the benefits of this calling convention?

\begin{solution}
Passing parameters in registers avoids memory access and reduces stack frame
sizes.
\end{solution}

\part[8]
How are calls handled by callers and by callees according to this convention?
Base your explanation on the given instructions.

\begin{solution}
The caller passes the first parameters in registers (\Verb+mov  AX 21+,
\Verb+mov DX 42+) and
pushs remaining parameters right-to-left on the stack (not in this
  example).
The callee saves the old base pointer on the stack (\Verb+push BP+) and
initialises the new one (\Verb+mov  BP SP+).
Next, it saves registers which it needs to preserve (not in this example).
Finally, it leaves the result in \Verb+AX+ (\Verb+add AX DX+),
restores the registers (not in this example) 
and the base pointer (\Verb+pop BP+),
and cleans the stack (not in this example),
before it returns (\Verb+ret+).

\end{solution}

\end{parts}

